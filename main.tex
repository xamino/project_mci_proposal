\documentclass{acm_proc_article-sp}

% UTF8 encoding and scalable fonts
\usepackage[utf8]{inputenc}
\usepackage[T1]{fontenc}

\usepackage{microtype}
\usepackage{graphicx}
\usepackage{subfigure}
\usepackage{booktabs}
\usepackage{listings}
\usepackage[hyphens]{url}
\usepackage[hidelinks]{hyperref}

\begin{document}

% Change to your title
\title{MIND – Media Informatics Networking Displays}

% Change to your personal details
\numberofauthors{4}
\author{
\alignauthor
Andreas Köll\\
       \affaddr{Institute of Media Informatics}\\
       \affaddr{Ulm University}\\
       \affaddr{Ulm, Germany}\\
       \email{andreas.koell@uni-ulm.de}
\alignauthor
Laura Irlinger\\
       \affaddr{Institute of Media Informatics}\\
       \affaddr{Ulm University}\\
       \affaddr{Ulm, Germany}\\
       \email{laura.irlinger@uni-ulm.de}
\alignauthor
Patryk Boczon\\
       \affaddr{Institute of Media Informatics}\\
       \affaddr{Ulm University}\\
       \affaddr{Ulm, Germany}\\
       \email{patryk.boczon@uni-ulm.de}
\alignauthor
Tamino P.S.M. Hartmann\\
       \affaddr{Institute of Media Informatics}\\
       \affaddr{Ulm University}\\
       \affaddr{Ulm, Germany}\\
       \email{tamino.hartmann@uni-ulm.de}
}

\maketitle
\begin{abstract}
The abstract should summarize the complete paper and give a specific overview of what is covered in the paper. Do not confuse the abstract with an introduction.
\end{abstract}

\section{Introduction}
Please have a look at the ACM sample file and FAQs at \url{acm.org/sigs/publications/proceedings-templates} for formatting help, concerning figures, tables, and math. Examples of Table and Figure environments are included as comments below.

\subsection{A Subsection}

\subsubsection{A subsubsection}
Usually three hierarchy levels are sufficient for papers. You should rethink your structure, if you need more levels. \cite{bowman:reasoning}

%\begin{table}
%	\centering
%	\caption{Frequency of Special Characters}
%	\label{tab:table1}
%	\begin{tabular}{|c|c|l|} \hline
%	Non-English or Math&Frequency&Comments\\ \hline
%	\O & 1 in 1,000& For Swedish names\\ \hline
%	$\pi$ & 1 in 5& Common in math\\ \hline
%	\$ & 4 in 5 & Used in business\\ \hline
%	$\Psi^2_1$ & 1 in 40,000& Unexplained usage\\
%	\hline\end{tabular}
%\end{table}
%
%\begin{table*} % use the table* version to create tables spanning both columns
%	\centering
%	\caption{Some Typical Commands}
%	\label{tab:table2}
%	\begin{tabular}{|c|c|l|} \hline
%	Command&A Number&Comments\\ \hline
%	\texttt{{\char'134}alignauthor} & 100& Author alignment\\ \hline
%	\texttt{{\char'134}numberofauthors}& 200& Author enumeration\\ \hline
%	\texttt{{\char'134}table}& 300 & For tables\\ \hline
%	\texttt{{\char'134}table*}& 400& For wider tables\\ \hline\end{tabular}
%\end{table*}
%
%\begin{figure}
%	\centering
%	\includegraphics[width=\columnwidth]{fly.pdf}
%	\caption{A fly}
%	\label{fig:fly}
%\end{figure}
%
%\begin{displaymath}
%\lim_{x\rightarrow\infty} \frac{f(x)}{g(x)} = L.
%\end{displaymath}

\bibliographystyle{abbrv}
\bibliography{sources}

% use this to make sure that columns are balanced on the last page.
\balancecolumns

\end{document}
