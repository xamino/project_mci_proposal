\section{Networking}

Networking plays a central part in many applications nowadays.
In the following we will take a quick look at the various aspects that can and should be considered when developing a system that partly or completely relies on networking.

\subsection{Wireless Connections}

The first to consider is the physical technology (specifically the electromagnetic spectrum) with which to enable the connection between different devices.
Three technologies have been widely adopted: wireless LAN, bluetooth, and infrared.
Wireless LAN, generally referred to as wifi, is the most ubiquitous technology.
It is broadly interoperable and a multitude of supported devices exist.
Due to its comprehensive use, a lot of research has been focused on it, allowing a very comprehensive understanding of its capabilities.
However the technology has a few significant drawbacks.
Wifi generally does not have an excellent range due to its spectrum allocation.
Any project that wants to use wifi nowadays for special communications will always have to share the medium with normal devices – meaning that there are often competing communications that congest channels.
Bluetooth is another relatively common technology, mostly for accessory functions between near devices.
However it has even lower range than wifi and doesn’t offer as high data rates.
Infrared communications is only useable for communications in a very limited manner.
It requires a direct line of sight to initiate a connection and offers even worse data transfer rates than bluetooth.
However it has a single significant advantage: interoperability with devices that aren’t as high-tech as those that utilize bluetooth or wifi, such as television remotes.

\subsection{Communication Hierarchy}

Once a connection is physically possible, one has to consider how to structure the communication hierarchy.
Principally two separate possibilities exist, although they can easily be combined to varying degrees.
Most common is the strictly hierarchical structure in the form of a tree graph.
This structure is defined by the fact that communications are always facilitated via a common higher-level node.
This form is how the current internet is structured.
The upsides are varied.
It allows a more centralized and streamlined integration for capabilities – meaning that to support new capabilities, only a single higher node must be upgraded – the lower nodes that rely on the service will not interfere with other higher nodes as they are strictly separated.
However this also poses a significant risk: if a higher node is disrupted, all connected lower nodes are equally affected.
That means that this form has a low resistance to failure.
Because communications between two low nodes has to pass up and back down the tree, communication paths can be unnecessarily long – such as when two devices in the same room send an email between each other.

The alternative is a mesh structure, or a graph where the nodes are connected to multiple surrounding nodes independent of their perceived authority.
This allows easy addition of new devices to the mesh of communications, even for nodes that offer central services.
It also allows a decentralization of network technology as dedicated routers and servers can be distributed onto every connecting device.
A mesh structure also has a high fault tolerance towards disrupted routes: if nodes are disrupted, communications can easily be rerouted via alternative routes.
However mesh networks come with some drawbacks that have proven to be non-trivial to solve.
Negotiating dynamic communication links has proven to be difficult to implement correctly, and is often linked with a drastic decrease in available data bandwidth.

Some interesting notes to network structure and the various fallacies that should be avoided can be found in \cite{lai2002network}.

\subsection{Transferring Data or Programs}

Something that can also be considered when thinking with networks is whether we want to transfer only data via the network or if we also want to transfer running applications.
Combinations of the two are also possible.
Currently however the internet is only used to transfer data – programs are transferred but as data.