\section{Related Work}
% Mit bold, jeder einen Absatz
The following section contains an overview of related work, categorized in the possible system types.
%The following section contains an overview of related work. First research about system types is collected. Afterwards an overview about possible input interactions is given, followed by related work about output. Lastly some papers about networking are granted.

\subsection{General}
\paragraph{Modeling multimodal human-computer interaction}
Obrenovic and Starcevic \cite{obrenovic_modeling_2004} observed modeling input and output modalities using UML2.
According to them input modalities can be categorized as streaming-based and event-based.
Event-based input modalities are for example input via a keyboard or mouse, streaming-based input is for example input from sensors (microphone etc.).
They also differentiate between two general types of output.
Static and dynamic output.
Static output is for example a picture, dynamic output a video.
Some forms of dynamic output therefore consist of animated static output.
Another topic they discuss is the response time of systems.
There is the socalled perceptual processing time (< 0.1sec) which appears to a person instantaneous and multiple such events as continuous.
There is the immediate response time (~1sec) which is the minimum time a person requires to react to a new situation and there is the unit task time (~10sec) which is the scale for the simplest tasks the person wants to perform.

\subsection{One-To-One}

\paragraph{Clearing the Virtual Window - Connecting Two Locations
with Interactive Public Displays }
Häkkilä et al. \cite{hakkila_clearing_2013} investigate notifying the other end for a live video feed connection between two locations via public displays.
Each window represents a frozen widow to the other room.
By using gestures the ice could be melted from both ends allowing visual communication.
In their study using a two-sided ice melting design was most comfortable for their participants as communication attempts could be seen but also the privacy was respected.

\subsection{One-To-Many}

\subsection{One-To-System}

\paragraph{Exploring the utility of remote messaging and situated office door displays}
Cheverst et al. proposed \textbf{remote messaging and situated office door displays} \cite{cheverst2003exploring}.
Dating back to a time when SMS was extremely prominent, this work investigates the utility of sending messages not to a particular cellphone owned by a person but to one displays situated at particular locations.
One purpose of such an implementation was to explore whether a digital alternative is an enhancement to traditional post-it notes.
The owner of a display can leave a message on the display itself, via a web interface, an email client or by SMS messages.
Messages left by visitors can be entered at the display.
Those messages, however, are not visible for visitors due to privacy and spatial reasons.
The owner can read messages on the display itself, through a web interface or email, thus rendering such an implementation as one-to-system or system-to-one system type.
An extension of visitors’ interaction with office displays is introduced by Cheverst et al. in \cite{cheverst2005exploring}, describing the usage of Bluetooth as an in- as well as output channel for visitors.
Additionally to leaving a message, visitors can download certain data from the display owner.
With the limited range of Bluetooth, the visitor has to approach the display to a few meters, which might be argued to be an inconvenience since data can be obtain using the internet, as well.
However, according to \cite{cheverst2005exploring}, owners of data are more willing to share (more private) data with visitors who have made the journey to their office.

\paragraph{Lean and zoom: proximity-aware user interface and content magnification}
Harrison and Dey \cite{harrison_lean_2008} use the user's proximity to a screen and their leaning pose to determine the size of the information to be displayed.
Leaning forward resulted in a proportional magnification of the screen's display.
In their study participants found that items on screen could be read easier an more comfortably using this technique.

\paragraph{Mobile applications for open display networks: common design considerations}
José et al. \cite{jose_mobile_2013} look at ways a user can interact with public displays using his mobile device.
This includes creating the association between mobile device and display, using the device as controller and transmitting data.


\subsection{System-To-One}

\paragraph{Ambient Displays: Turning Architectural Space into an Interface between People and Digital Information}
Wisneski et al. \cite{wisneski_ambient_1998} create an ambient room and take a look at what possible mapping techniques there are to represent different types of information to the user. They use light, sound and movement to inform the user in a subtle way.

\subsection{Many-To-Many}

\paragraph{The Interacting Places Framework: Conceptualizing Public Display Applications That Promote Community Interaction and Place Awareness}
Memarovic et al. \cite{memarovic_places} describe potential \textbf{public display} applications to foster community interactions.
This work considers applications for a public display scenario and proposed a framework for implementation purposes.
Potential applications comprise e.g. transmitting text or (Flickr) images to a public display to be shown.
In addition, social media content can be displayed such as twitter feeds or facebook sites.
Aside from that, content can be service provided and does not have to originate from users.
This includes bus schedules, advertisement or other location based information such as the weather forecasts or upcoming events.
Such a mode can be described as a system-to-one or system-to-many system type with the output being initiated by the system itself.
Among the possibilities for producing the to-be-displayed content are the publishment of information via smartphone, email, social media or a simply by dropping the content in a network folder.

\paragraph{Screen Codes: Visual Hyperlinks for Displays}
In order to distinguish multiple interactions on public displays, Collomosse and Kindberg \cite{Collomosse_ScreenCodes} proposed C-Blink as visual-based interaction channel. The screen of the user’s smartphone emits a sequence of colors which represents coded information. The sequence is seized by a camera mounted on the public display, thus encoding the information.

\paragraph{Yarely: a software player for open pervasive display networks}
Yarely is a software player designed by Clinch et al. \cite{Clinch_Yarely} which is based on pervasive display networks. Whereas conventional media players are controlled by one authority, Yarely is supposed to perceive content as well as a play out schedule from several sources simultaneously. The actual output of content will be evaluated by the player itself.
One design goal of Yearly was extensibility. The integration of new components, e.g. adding environmental sensing for input purposes or rendering of the displayed content, is intended to cause minimal changes to the core player.


\subsection{System-To-Many}

\paragraph{Crossmodal Attention in Public-Private Displays}
Olivier et al. \cite{Olivier_Crossmodal} propose CrossBoard, where the user’s smartphone is synchronized with the public display.
In order to distinguish information that is relevant for the user, the display highlights pieces of information periodically while the user’s smartphone vibrates as long as the relevant information is highlighted.

\paragraph{Interactive Public Ambient Displays: Transitioning from Implicit to Explicit, Public to Personal, Interaction with Multiple Users}
Vogel et al.\cite{Vogel_InteractivePublicAmbient} exploit body tracking to determine which information should be displayed on which section of big public screens. Position (proximity to the screen) as well as posture of the user indicate the size and position of the relevant section of the display while RFID tags or face recognition is applied for identification and selection of information.

\paragraph{Visual Highlighting on Public Displays}
Ostkamp et al. \cite{Ostkamp_VisualHighlighting} introduce AR-Multipleye as an AR-based approach for visual highlighting using smartphones. A QR tag is situated on the public screen and provides information about different visual highlightings. A hint for the relevant information may e.g. be a transparent overlay on the piece of information when seen through the smartphone’s camera.

% many to system <-> system to many?
\paragraph{Proxemic Interaction: Designing for a Proximity and Orientation-aware Environment}
Ballendat et al. \cite{ballendat_proxemic_2010} take the concept of proxemics further and develop a proxemic media player which reacts to the angle, movement and view of one or multiple persons in front of a ambient display.
So looking away from the monitor pauses the movie, moving towards the display shows more information the closer the person is to the display.
E.g. from far away an overview of running movies is shown.
From closer up also information like its title is shown.
Sitting on the couch toggles the movie fullscreen.
A second user approaching the display is shown the title when away and a detailed information in a split screen view when close to the display.
The second view continues playing the movie.

\paragraph{Ambient displays and mobile devices for the creation of social architectural spaces}
Streits et al. \cite{streitz_ambient_2003} show a public display called Hello.Wall to show information about people in it's proximity.
They differentiate between Interaction, Notification and Ambient Space which allow different interaction with the display.
This information is also used to be transmitted to a remote identical display and thus enhancing people in a remote place discretely about knowledge of the other location.
This knowledge includes the general mood obtained from mobile devices (View.Port) associated with each person and can be used to see if there is the will for a video conference.
The mobile device also functions as private output when within interaction zone of the display.
Private information is otherwise coded in a abstract symbol on the display itself notifying the user associated with it about its existence.




\subsection{Input}
Nacenta et al. \cite{a18-nacenta} introduced a multi-user system called LunchTable.
Often people in groups check information on their mobile devices during conversations (for example during lunch). But for sharing information in groups mobile devices are less suitable: displays are simply too small.
Large displays are easily shared by multiple users. So interaction in groups shall be simplified by a multi-user multi-display system: the LunchTable.
The prototype uses a vertical display on the wall and a multi-touch display embedded within a regular lunch table.
The information on the display can be controlled via the multi-touch display.
The users can manipulate the wall display space by interacting with the horizontal surface window manager (LunchTable).
Windows, also several simultaneously, can be manipulated by dragging their thumbnail. Moving application tiles into the regions of the window manager results in opening them on the display.
For a simple interaction with the display there are virtual input devices on the lunch table (trackpad and keyboard).
They can be opened with a trackpad icon and are connected to the thumbnail of the application.

Boring et al. \cite{p161-boring} used the mobile phone as an input device for large public displays.
They introduced three different interaction techniques: Scroll, Tilt and Move. With that the user can continuously control a pointer located on a remote display.
To move the pointer on the large display a user can press the joystick on the phone in the desired direction (Scrolling).
A user can accelerate the pointer by tilting the phone in the favored direction (Tilting).
Lastly the mobile phone can be used like an interactive laser-pointer (Moving).
Bidirectional interaction (input and output together) is possible using these techniques.
A study showed Move and Tilt to be faster but more error prone techniques for selection tasks compared to Scroll.

According to Baldauf et al. \cite{a4-baldauf} a camera-based control of large markerless displays through smartphones is a novel interaction technique.
To enable this for a large network of screens they implemented a prototype framework.
Hence the smartphone has to be able to register itself at one display (for example with a wireless connection).
So input commands can be communicated.
Afterwards the display and the smartphone periodically exchange image information.
Such the relation between the display and the mobile phone can be defined and so touch interactions on the mobile phone can be executed.


\subsection{Privacy}

This subsection lists related work concerning to privacy aspects of displays.
This will include public displays and display networks.

One way of allowing privacy on a single display is by restricting the content that can be viewed into angled field of views.
This is useful in the instance where a shared horizontal display is being utilized, such as a display table.
Whenever multiple users are present, viewing private data is not possible without making it visible to others.
One way of allowing the separation of private and public data can be achieved by a display mask that allows a single display to serve multiple views, as seen in \cite{smith2008public}.
In the paper, the authors present a working mask that allows public and private data to be shown together on a per user basis.
Public data is simply shown on all separate views (copying it to their framebuffers), while private data remains viewable to only the single originating instance.

Another interesting way of handling private and public aspects is found in \cite{baldauf2012private}.
Here, the public display serves as a common selection interface, while mobile devices are used to view the selected videos in a private way.
Baldauf et al. note that smartphones are now a common companion and thus well suited to be used as a private controller and display for interaction with public systems.
The paper also states that using the smartphone display as the private output was preferred over a competitive mode, where control of the public display was contested by the various users for showing a video.

\cite{shoemaker2001single} points out that one of the problems of using personal devices as the personal space is that the user has to divide her attention between multiple devices.
It is stated that this results in unnecessary cognitive overload for the user.
The paper shows a system that allows private information to be shown on a public display by utilizing shutter glasses and separating the content on the public display by timing the frames to the respective user's glasses.
The authors found the method to work well and to have a good acceptance.
Some of the questions raised include the question whether it is important to support private work in a collaborative setting; and how to signal to the user which data object is private and which is publicly viewable.

\subsection{Networking}

Ian F. Akyildiz et al. \cite{akyildiz2005survey} give a good introduction to \textbf{mesh networks}.
They highlight where such networks might be useful, the advantages of using meshes, and also note the problems that still need to be solved.
The two main hurdles to overcome are robust scalability and security.
Scalability is required for a large number of mesh devices; security is important because the mesh devices shouldn't be trusted as a user has no control over them – they are not a trusted third party.
However since the above paper has been published, new research and new implementations have made significant steps towards solving some of these problems.
One example of such a possible solution is the Hyperboria \cite{hyperboria} project.

Some good questions pertaining to networking for \textbf{multiple display networks} are posed by Ferreira et al. \cite{ferreira2012scalability}.
The question, whether the same web standards will apply, and whether display networks will affect network performance should be considered in future work.

The survey by L. Atzori et al. \cite{atzori2010internet} concerns itself with the so-called \textbf{internet of things}.
In it the authors note that any advancement for an internet of things (henceforth IoT) must come from a wide range of topics, such as telecommunications or social sciences; this is also the case for our project.
Another aspect that might be of note is that the IoT includes an environment where smart objects interact with each other, much like computers do via the internet.
The survey highlights that a possible next step for the development of networks is away from the virtual space to a more physical one.
However, a range of technological and social issues will need to be addressed before that becomes a feasible future.
The main issue as stated are scalability and efficiency.
