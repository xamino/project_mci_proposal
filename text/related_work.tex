\section{Related Work}
% Mit bold, jeder einen Absatz

\subsection{System Types}
Service modes/System types and applications
A wide range of various system types (/service modes) occurs in the field of pervasive displaying.

Memarovic et al. \cite{memarovic_places} describe potential \textbf{public display} applications to foster community interactions.
This work considers applications for a public display scenario and proposed a framework for implementation purposes.
Potential applications comprise e.g. transmitting text or (Flickr) images to a public display to be shown, matching a one-to-one or one-to-many system type.
In addition, social media content can be displayed such as twitter feeds or facebook sites.
Aside from that, content can be service provided and does not have to originate from users.
This includes bus schedules, advertisement or other location based information such as the weather forecasts or upcoming events.
Such a mode can be described as a system-to-one or system-to-many system type with the output being initiated by the system itself.
Among the possibilities for producing the to-be-displayed content are the publishment of information via smartphone, email, social media or a simply by dropping the content in a network folder.

Cheverst et al. proposed \textbf{remote messaging and situated office door displays} \cite{cheverst2003exploring}.
Dating back to a time when SMS was extremely prominent, this work investigates the utility of sending messages not to a particular cellphone owned by a person but to one displays situated at particular locations.
One purpose of such an implementation was to explore whether a digital alternative is an enhancement to traditional post-it notes.
The owner of a display can leave a message on the display itself, via a web interface, an email client or by SMS messages.
Messages left by visitors can be entered at the display.
Those messages, however, are not visible for visitors due to privacy and spatial reasons.
The owner can read messages on the display itself, through a web interface or email, thus rendering such an implementation as one-to-system or system-to-one system type.
An extension of visitors’ interaction with office displays is introduced by Cheverst et al. in \cite{cheverst2005exploring}, describing the usage of Bluetooth as an in- as well as output channel for visitors.
Additionally to leaving a message, visitors can download certain data from the display owner.
With the limited range of Bluetooth, the visitor has to approach the display to a few meters, which might be argued to be an inconvenience since data can be obtain using the internet, as well.
However, according to [], owners of data are more willingly to share (more private) data with visitors who have made the journey to their office.

gia input

\subsection{Output}
Obrenovic and Starcevic \cite{obrenovic_modeling_2004} take a look at modeling input and output modalities using UML2. They differentiate between two general types of output. Static and dynamic output. Static output is for example a picture, dynamic output a video. Some forms of dynamic output therefore consist of animated static output. Another topic they discuss is the response time of systems. There is the socalled perceptual processing time (< 0.1sec) which appears to a person instantaneous and mutliple such events as continous. There is the immediate response time (~1sec) which is the minimum time a person requires to react to a new situation and there is the unit task time (~10sec) which is the scale for the simplest tasks the person wants to perform.


Streits et al \cite{streitz_ambient_2003} show a public display called Hello.Wall to show information about people in it’s proximity. They differentiate between Interaction, Notification and Ambient Space which allow different interaction with the display. This information is also used to be transmitted to a remote indentical display and thus enhancing people in a remote place discretly about knowledge of the other location. This knowledge includes the general mood obtained from mobile devices (View.Port) associated with each person and can be used to see if there is the will for a video conference. The mobile device also functions as private output when within interaction zone of the display. Private information is otherwise coded in a abstract symbol on the display itself notifying the user associated with it about its existence.


Ballendat et al \cite{ballendat_proxemic_2010} take the concept of proxemics further and develop a proxemic media player which reacts to the angle, movement and view of one or multiple persons in front of a ambient display.
So looking away from the monitor pauses the movie, moving towards the display shows more information the closer the person is to the display. E.g. from far away an overview of running movies is shown. From closer up also information like its title is shown. Sitting on the couch toggles the movie fullscreen. A second user approaching the display is shown the title when away and a detailed information in a split screen view when close to the display. The second view continues playing the movie.



\subsection{Networking}

Ian F. Akyildiz et al. \cite{akyildiz2005survey} give a good introduction to \textbf{mesh networks}.
They highlight where such networks might be useful, the advantages of using meshes, and also note the problems that still need to be solved.
The two main hurdles to overcome are robust scalability and security.
Scalability is required for a large number of mesh devices; security is important because the mesh devices shouldn't be trusted as a user has no control over them – they are not a trusted third party.
However since the above paper has been published, new research and new implementations have made significant steps towards solving some of these problems.
One example of such a possible solution is the Hyperboria \cite{hyperboria} project.

Some good questions pertaining to networking for \textbf{multiple display networks} are posed by Ferreira et al. \cite{ferreira2012scalability}.
The question, whether the same web standards will apply, and whether display networks will affect network performance should be considered in future work.
