\section{System Types}

In the following we will take a look at the possible system, application, and human interaction possibilities.
We will evaluate and categorize the different system to person combinations for their usability as a human-computer-interaction project.

Regarding a multi-user system, there are different entity-relationships to be considered, each with its own concomitant features.
In the following, the entities described in the relationships are the intended endpoints or terminal devices of the connections.

Among the most common relationships is the link between the user and the system, short \textbf{one-to-system}.
This connection implies e.g. a user storing data, such as appointments in her/his calendar, to be accessible afterwards.

A further relationship is represented by the link between two users, hence \textbf{one-to-one }relationship.
This configuration is commonly used for transferring data between two users or alternatively two devices.
Such a communication can provide a high level of privacy, provided each device represents one user, e.g. when using smartphones.

The \textbf{one-to-many} configuration represents a scenario in which, starting from one terminal device, a broadcast is sent to a pre-defined compilation of devices.
This phenomenon is omnipresent in respect of social networking, thus rendering this configuration extremely common - with all its concomitant privacy issues.

The \textbf{system-to-one} link describes a relationship where the interaction is invoked by the system and provokes output on one terminal device.
This can be implemented pre-setting a date and time, registering a certain change in the environment using sensors or triggering a notification message to be sent after finishing a computing task.
In an extreme case, this configuration doesn’t require any interaction with the user, however the output may show a deficit in the matter of privacy, for instance when the user receives private information while not interacting with the device.
This effect may even be enhanced when the information is sent to a more or less public display in a network of displays.

The \textbf{system-to-many} relationship can be rated similar to the system-to-one configuration, with an even greater loss in privacy - since potentially confidential information might be received by multiple devices without the proper recipient on site.
Possible applications involve reminders for group activities and clues about changes in the environment, such as emergency notifications.

A \textbf{many-to-many} relationship ordinarily applies in network gaming or (video) conferencing. Although privacy issues may arise from such a configuration, it enables simultaneous interaction of multiple users.