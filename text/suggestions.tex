\section{Suggestions}

This section will briefly highlight possible ideas to utilize with a display network.
No selection has been made yet.
This serves simply for completeness.

\subsection{Flexible Application Runtime Environment}

One proposal is a framework for a runtime environment where applications can be fluidly transferred between different devices.
This would include private machines such as a personal laptop or smartphone, but also public devices such as a projector.
The general idea is that instead of simply allowing data transfer through a network, FARE would allow applications to switch devices during runtime, as seamlessly as possible.

Such a framework would decrease the dependence of programs on a single machine.
Instead, programs would run independent of hardware.
This would allow some very interesting solutions: for example, FARE could allow applications to automatically stay near to a user while the user is moving – either by transferring to a smartphone when the user exits her home or by jumping to public displays when allowed to.

As security is paramount for data and applications in our time, it has to be built into FARE from the beginning.
This could be achieved by dividing devices into categories by administrators.
Two such categories might be private and public.
The private category would be for any devices that the user absolutely trusts.
On these the applications would be freely transferable through a simple swipe of the window to an edge.
Remote pull might also be allowed, dependent on user choice.
The public category would be for multi-user displays and or projectors.
These would allow an application to access them with a sub-task.
An example for this would be a presentation creation software that would allow a presentation sub-task to be temporally moved to the public display, but be controlled from the creation software on the home device.
Such public devices might also allow the easy copying of software and associated data for collaborative tasks.

Generally speaking, the framework could be extended to be location aware, user aware, and incorporate external stimuli such as smart artifacts.
Care would have to be taken however to ensure that programs fail softly if required hardware is not available.
FARE would allow user interaction with machines to move away from programs back to the data, as it should be.
A data oriented computer platform would decouple the dependency of programs from their data, thus allowing an increase in data usage flexibility.

Due to the aspect that this suggestions is a technical framework and not a directly usable application, it has been discarded.
However, some aspects might still be implemented in the final project as deemed necessary or helpful.

\subsection{Ambient display office}
In an office where a person's location can be tracked, and also his proximity to a display, this information can be shared with other displays in the network and used for collaborative work.
The person is identified either using CV or by logging in automatically with his mobile device.
The display itself shows an abstract non-intrusive overview of all rooms and the persons identity and location are coded symbolically.
Also the display adapts to the user's distance from it.
Is he far away, only general information or alerts are shown.
In closer proximity more details follow. Directly infront of it persons have the ability to interaction with the display.
But not only does the display provide information about a person's location, but also his intentions.
These are controlled using his mobile device and for example determine his will to participate in a video conference with a specific room.
If such intentions are shared among two or more office rooms, the system initiated the video conference or asks the persons for it.

\subsection{Office aware display network}
..

\subsection{Augmented Office}
For a network of displays endowed with cameras, situated in each office, an extension of the office(s) via augmented reality is thinkable.
Users can augment their offices by adding a schedule and placing virtual shelves, clipboards etc. into their offices - each corresponding to a different topic e.g. corresponding to research topics, theses, courses.
The augmentation is based on the camera's picture.
Multiple users in one office are possible as well, since each user has his own augmented equipment which does not interfere with other users' placed furnishing.

Visiting users can put files, links, messages etc. into/onto the augmented areas via their own display interface or e.g. a network interface.
The level of urgency and publicity can be set at the posting process.
Postings or data annotated as public can be viewed by other visitors.
The recipient gets a notification for postings on his display and for example on his smartphone.
In the case of absence or a turned off camera (e.g. for privacy reasons) users can see a default picture of the recipient’s office with up-to-date augmented features and schedule of the recipient.

Such a system would relive the confusion which exists with a standard email-based exchange of information by pre-categorizing future incoming information and data.
Additionally, sending broadcasts and messages in general of high priority to be recognized immediately can be achieved using Augmented Office.
Finally, sending and accepting invites for applications which are directly performed on the displays seem also to be an appropriate approach for such a system. 

