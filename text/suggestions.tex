\section{Suggestions}

TODO: Einleitungstext

\subsection{Flexible Application Runtime Environment}

One proposal is a framework for a runtime environment where applications can be fluidly transferred between different devices.
This would include private machines such as a personal laptop or smartphone, but also public devices such as a projector.
The general idea is that instead of simply allowing data transfer through a network, FARE would allow applications to switch devices during runtime, as seamlessly as possible.

Such a framework would decrease the dependence of programs on a single machine.
Instead, programs would run independent of hardware.
This would allow some very interesting solutions: for example, FARE could allow applications to automatically stay near to a user while the user is moving – either by transferring to a smartphone when the user exits her home or by jumping to public displays when allowed to.

As security is paramount for data and applications in our time, it has to be built into FARE from the beginning.
This could be achieved by dividing devices into categories by administrators.
Two such categories might be private and public.
The private category would be for any devices that the user absolutely trusts.
On these the applications would be freely transferable through a simple swipe of the window to an edge.
Remote pull might also be allowed, dependent on user choice.
The public category would be for multi-user displays and or projectors.
These would allow an application to access them with a sub-task.
An example for this would be a presentation creation software that would allow a presentation sub-task to be temporally moved to the public display, but be controlled from the creation software on the home device.
Such public devices might also allow the easy copying of software and associated data for collaborative tasks.

Generally speaking, the framework could be extended to be location aware, user aware, and incorporate external stimuli such as smart artifacts.
Care would have to be taken however to ensure that programs fail softly if required hardware is not available.
FARE would allow user interaction with machines to move away from programs back to the data, as it should be.
A data oriented computer platform would decouple the dependency of programs from their data, thus allowing an increase in data usage flexibility.

% Notes: pull might also be done via a link per email, for example, allowing easy sharing for applications by people not near each other – for example making a presentation available to someone in another country. Applications would be coupled to data instead of the other way around, as is nowadays paramount. This would mean that each device always only has to have the programs on it for the data on it – if a new type of data is received, the application will automatically be bundled with it. If an application already exists, only the data has to be transferred.
% TODO: Integrate idea with display network in the explanation – you have to sell it to Christian. :D (display network --> easy networking of protected / public display.)

\subsection{Smart interactive office}
Andi's ideas yeah!